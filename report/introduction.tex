\section{Introduction}
\label{introduction}
Authorship Attribution is the science of attributing the ownership of a disputed or anonymous text, given a list of potential authors. A corpora (a relevant subset of these authors texts) are then analysed, and the gained information is then used to find the most likely author of the disputed text.\\

The most famous case of Author Attribution is the attribution of 12 essays written in the \q{The Federalist Papers} series, whom both Alexander Hamilton and James Madison claimed to have written. Most modern Authorship Analysis believes that James Madison wrote them \cite{Fung03thedisputed}\fixme{Tentive: Is this good enough?}.\\
 The introduction of computers, and the Internet, where texts can be published anonymously, has increased the importance of Author Attribution, as it is now easier than ever before to commit plagiarism or illegal sharing of texts (such as for school projects). Author Attribution might also be used for the purpose of Internet related harassment, or inflammatory speech - thus circumventing the otherwise implicit anonymity found on the Internet \fixme{should it be here? Is it correctly worded}.\\ 

I will in this paper try to apply an already developed Author Attribution technique to posts found on Internet forums, and try to evaluate the quality of the result.\\

Internet forums are interesting to study, since the posts that comprise them are short, informal, and full of bad spelling and grammar\footnote{Not to mention the many strange abbreviations (``lol'', ``btw''), and features unique to the Internet (such as hyperlinks or embedded pictures)}. All of this quite unlike the works of famous authors (who tend to be used in the studies \cite{nr4}). 

\subsection{Scope and Limitations}
\label{scope}
A common feature in papers about Authorship Attribution is that their sample languages tend to be English, Greek and Chinese \cite{syntactic}, \cite{nr2}, \cite{nr4} and \cite{app-spe}. Since I do not have any relation to any Greek or Chinese forum, and indeed no ability to speak or understand either language, I have chosen to exclusively concentrate on forums and methods that works on a character based language - more specifically Danish.

\subsection{Expectations to the reader}
\label{expectations}
I expect the reader to be familiar with the problems within the field Author Attribution, as well as having a theoretical and practical experience with applying these techniques - especially the n-gram and statistical methods. Due to the nature of the sample content, it would be advisable if the reader has a familiarity with a cross section of relaxed non-formal Internet forums, and the kind of posts made on these. Sites such as Slashdot.org or a console gaming forum would be preferable.

\subsection{Learning targets and objectives}
\label{learning}
\subsubsection{Learning targets}
After having completed this assignment I will have learned 
\begin{itemize}
\item To briefly be able to summarise techniques, practical as well as theoretical, for authorship attribution.
\item Apply a simple technique for authorship attribution on a limited and specific domain with texts, which have special syntactical characteristics.
\item To be able to implement and test the above in a practical manner.
\item To be able to properly interpret these test results
\end{itemize}

\subsection{Reader guide}
In this section I give a quick overview of what the different section in report will cover.
\fixme{Final fixme: Make sure that it points to the correct sections}
\begin{description}
\item[Section \ref{introduction}] The introduction to the report, which will introduce the subject and explain my goals
\item[Section \ref{choiceMethod}] A survey of the different methods to do Author Attribution  
\item[Section \ref{method}] Explanation of the method used in my project
\item[Section \ref{consideration}] Considerations 
\item[Section \ref{tests}] Tests
\item[Section \ref{interpretation}] Interpretation of results
\item[Section \ref{conclusion}] The conclusion of the report, where I will sum up the findings  
\end{description}

