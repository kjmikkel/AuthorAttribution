\section{Introduction}
\label{introduction}
Authorship Attribution is the science of assigning ownership of a disputed or anonymous text, given a list of potential authors. A relevant subset of these authors' texts, a so-called corpus, is analysed, and the gained information is then used to find the most likely author of the text.

The most famous case of Author Attribution is the attribution of 12 essays written in the ``The Federalist Papers'' series, whom both Alexander Hamilton and James Madison claimed to have written. Many modern works on Authorship Analysis attribute the essays to James Madison \cite{Fung03thedisputed}.

The introduction of computers and the Internet, where texts can be published anonymously, has increased the importance of Author Attribution, as it is now easier than ever to commit plagiarism, e.g. such as for school projects. Authorship Attribution might also be used on Internet forums and message boards, where it can be applied to root out regulars posting either under a false name (the so called \emph{sock puppets}) or anonymously.

I will in this report apply an already developed Author Attribution technique to posts found on a Internet forum, and evaluate the quality of the technique.

Internet forums are interesting to study, since the posts that comprise them are short, informal, and usually contains bad spelling and grammar\footnote{Not to mention features unique to the Internet (such as hyperlinks and embedded videos and pictures)}. Posts on the Internet are thus quite unlike the works of famous authors --- who tend to be used in the studies like \cite{nr4} and \cite{nr2}.  

\subsection{Applications}
When applying Authorship Attribution techniques to texts on the Internet, it is important to consider in which contexts it is applicable. Ideally, we would like to use Authorship Attribution on all text on the Internet, in order to identify anonymous posters, presumed sock puppets or previously banned users. I find the most interesting of these to be the Forums/Messageboards, comment sections on large public websites and wikis. However, all of these except forums suffers from fundamental problems that prohibits a general method of being practical:

\begin{description}
\item[Public websites with comment features:] In services with comment features --- such as blogs, on-line newspapers or video streaming services there may be problems with the author to post ratio. Depending on how debatable the topic is, most authors will only post a few comments per item --- but if the item is popular, then this might be done by many authors, which generates a lot of posts. This means that for any given time block, there will be a large author to post ratio (i.e. close to 1) which necessitates that the comments for many items/threads would have to be collected in order to assemble a decent corpus for any given author. Forums and message boards does not suffer from this to the same degree, as almost all threads are created to start a debate. Forums and message boards also tends to have a greater social dimension, which means that authors might be more likely to return to the forum, unlike comments, where an author is more likely to comment just once.

\item[Wikis:] One of the problems that wikis face is vandalism, either in the form of erroneous material, or outright defacement of its pages --- an Author Attribution system would be helpful, as malicious users could be banned before they acted on their impulses. In practice, however, the following problems makes it practically infeasible:
\begin{description}
\item[Vandals:] Since an account handle is easy to create, or perhaps not even needed, the vandal can just register for a new one, and then vandalise, delete and deface pages he wants to. Since most wiki software allows the pages to be restored in a few clicks, and has provisions to ban the IP-address\footnote{Which acts as a far better identifier than an on-line handle} from which the changes took place, Author Attribution seem like an expensive approach.  

\item[Editing problem:] Another problem is that while a lot of authors will write long, connected, passage of texts\footnote{For instance the creator/main contributor of the article}, many other edits are minor changes that occur inside already existing passages --- which is hardly representative for the authors' corpus profile, and would have to be filtered out.   
\end{description} 
\end{description}

I therefore believe that forums and message boards are both the most feasible and on-line systems that Author Attribution can be applied to.

\subsection{Scope and Limitations}
\label{scope}
A common feature in papers about Authorship Attribution is that their sample languages tend to be English, Greek and Chinese \cite{syntactic}, \cite{nr2}, \cite{nr4} and \cite{app-spe}. Since I do not have any relation to any Greek or Chinese forum, and do not understand either language, I have chosen to concentrate exclusively on forums and methods that works on the Latin alphabet --- more specifically Danish.

\subsection{Expectations to the reader}
\label{expectations}
I expect the reader to be familiar with the field of Authorship Attribution, as well as having a theoretical and practical experience with applying Authorship Attribution techniques --- especially the $n$-gram and statistical methods. Due to the nature of the sample content, it would be advisable if the reader has a familiarity with a cross section of relaxed Internet forums, and the kind of posts made on these. Sites such as slashdot.org or a console gaming forum would be preferable.

\subsection{Learning targets and objectives}
\label{learning}
\subsubsection{Learning targets}
After having completed this assignment I will have learned 
\begin{itemize}
\item To be able to summarise techniques, practical as well as theoretical, for Authorship Attribution.
\item To apply a simple Authorship Attribution method on a limited and specific domain with texts.
\item To be able to implement and test the above in a practical manner.
\item To interpret the test results produced by this implementation.
\end{itemize}

\subsection{Terminology}
In this report I will use the words ``post'' and ``text'' interchangeably\footnote{Since each post is self-contained, I think it is valid to identify it as a short text}, and likewise with the words ``forum'' and ``message-board'', as there in this case is no practical distinction.

\subsection{Overview}
In this section I give a quick overview of what the different sections in report will cover.
\begin{description}
\item[Section \ref{introduction}] The introduction to the report, which will introduce the subject and explain my goals
\item[Section \ref{choiceMethod}] A survey of the different methods for Author Attribution  
\item[Section \ref{method}] Explanation of the Authorship Attribution method used in my project
\item[Section \ref{implementation}] Description of the implementation details and an analysis of the scalability of the implementation
\item[Section \ref{considerations}] The considerations that went into designing the various tests needed to gauge the effectiveness of the Authorship Attribution method found in section \ref{method}.
\item[Section \ref{tests}] The test results
\item[Section \ref{interpretation}] Interpretation of results
\item[Section \ref{conclusion}] The conclusion of the report  
\end{description}

\subsubsection*{Acknowledgements}
I would like to thank Johan Sejr Brinch Nielsen for giving me the forum data that I have been analysing.
