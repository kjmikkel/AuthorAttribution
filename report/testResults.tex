\section{Tests}
\label{tests}

In this section present my test results in order to determine the effectiveness of my implementation.

In order to make my test results comparable with \cite{nr4} I will use the same parameters that they used. To make it easier to compare with the results in \cite{nr4}, I will keep all percentage results as a faction (e.g. 1 would be 100 \% and 0.33 would be 33 \%) --- results not computed in \cite{nr4} will however, be in percentage. These I will look for are: 

\begin{description}
\item[Overall Average] The number of correctly attributed texts divided by the number of texts that have been attributed. E.g. if 8 texts have been correctly attributed, and 10 texts have been attributed, then the overall average would be 0.8 

\item[Precision] The number of correctly attributed texts for an author, divided by the total number of texts that have been attributed to that author. E.g. if the author has had 8 texts correctly attributed to him, and in total 10 texts have been attributed to him, then the Precision would be 0.8. Thus, this is essentially local accuracy.

\item[Recall] Recall is the number of correctly attributed texts for an author, divided by the by the number of texts that should have been attributed to. E.g. if the authors has had 8 texts correctly attributed to him, but there was 10 texts there was to be attributed --- then the recall is 0.8.

\item[F-measure] F-measure is calculated like so: $\frac{2 \times precision \times recall}{precision+recall}$ 

\item[Macro-average F-measure] The Macro-average F-measure is the average of all the authors F-measure. 
\end{description}

\subsection{Random selection}
In order to have a baseline to compare my results against, I have for each subtest made a random selection test. For each text that had to be selected in the test, I have selected one million random authors as the texts author. The result of this will be presented at the end of each subsection, and will be compared to the real results in section \ref{interpretation}. If the random selection returns better results than the method found in \cite{nr4}, then my implementation of \cite{nr4} cannot be used. 

Since I do a million random selections, the concept of precision will become meaningless in tests when there is only text from one author --- since there in practice is no chance of that author not being chosen once. I will therefore only show precision for Random selection tests when there are multiple authors. 

The code that generates the random selections can be found in section \ref{code:randomTest}.

\subsection*{Key}
In the following I each author will follow the following format: Axx$^{n}$, where xx is the number identifying the author, and n is the number of posts that the author has written. Furthermore, since no test contain every author, the authors who appear both in the corpus and the texts that are to be attributed will have the following formatting: \aAuthor{Axx$^{n}$}, whereas authors who only appear in the corpus will have no such formatting. All authors who have written only 1 post, will furthermore have the following formatting: \veryFew{Axx$^{n}$}. These formatting may overlap.\\

\subsection{Test results}

\subsubsection{StressTest1}
A single post is compared against the entire corpus --- for more information, \nref{StressTest1}.\nl

\begin{tabular}{|c||c|c||c|}
\hline 
\multicolumn{4}{|c|}{Computer Estimate}\\
\hline 
True Label & \aAuthor{A4}$^{137}$ & \veryFew{A66}$^{1}$ & Recall \\
\hline 
\aAuthor{A4}$^{137}$ &  & 1 &  0.0\\
\veryFew{A66}$^{1}$ &  &  &  0.0\\
\hline 
Precision & 0.0 & 0.0 & \\
\hline 
\multicolumn{4}{|c|}{Overall Accuracy: 0 Macro-average F-measure: 0.0}\\
\hline 
\end{tabular} 
\nl

\begin{tabular}{|c||c|c||c|}
\hline 
\multicolumn{4}{|c|}{Computer Estimate}\\
\hline 
True Label & \aAuthor{A38}$^{5}$ & \veryFew{A61}$^{1}$ & Recall \\
\hline 
\aAuthor{A38}$^{5}$ &  & 1 &  0.0\\
\veryFew{A61}$^{1}$ &  &  &  0.0\\
\hline 
Precision & 0.0 & 0.0 & \\
\hline 
\multicolumn{4}{|c|}{Overall Accuracy: 0 Macro-average F-measure: 0.0}\\
\hline 
\end{tabular} 
\nl

\begin{tabular}{|c||c|c||c|}
\hline 
\multicolumn{4}{|c|}{Computer Estimate}\\
\hline 
True Label & \veryFew{A50}$^{1}$ & \aAuthor{A65}$^{10}$ & Recall \\
\hline 
\veryFew{A50}$^{1}$ &  &  &  0.0\\
\aAuthor{A65}$^{10}$ & 1 &  &  0.0\\
\hline 
Precision & 0.0 & 0.0 & \\
\hline 
\multicolumn{4}{|c|}{Overall Accuracy: 0 Macro-average F-measure: 0.0}\\
\hline 
\end{tabular} 
\nl

\texttt{Random selection:}\nl
\begin{tabular}{|c|c|c|c|}
\hline 
Name & Recall &  Hits & Miss \\ 
\hline 
\aAuthor{A4$^{137}$} & 0.10 & 0.10 & 0.0 \\ 
\hline 
\multicolumn{4}{|c|}{Overall Accuracy: 0.104  Macro-average F-measure: 0.188 }\\ 
\hline 
\end{tabular}

\subsubsection{StressTest2}
A single post from an author who has written \postAmount{Some} posts are checked against the entire corpus, \nref{StressTest2}.\\

\texttt{Stress Test 2.1:}\nl
\begin{tabular}{|c||c||c|}
\hline 
\multicolumn{3}{|c|}{Computer Estimate}\\
\hline 
True Label & A4 & Recall \\
\hline 
A4 & 1 &  0.00\\
\hline 
Precision & 1.0 & \\
\hline 
\multicolumn{3}{|c|}{Overall Accuracy: 1.0 Macro-average F-measure: 0.333333333333}\\
\hline 
\end{tabular} 
\nl

\texttt{Stress Test 2.2:}\nl
\begin{tabular}{|c||c|c||c|}
\hline 
\multicolumn{4}{|c|}{Computer Estimate}\\
\hline 
True Label & \aAuthor{A0}$^{75}$ & A24$^{71}$ & Recall \\
\hline 
\aAuthor{A0}$^{75}$ &  & 1 &  0.0\\
A24$^{71}$ &  &  &  0.0\\
\hline 
Precision & 0.0 & 0.0 & \\
\hline 
\multicolumn{4}{|c|}{Overall Accuracy: 0 Macro-average F-measure: 0.0}\\
\hline 
\end{tabular} 
\nl

\texttt{Stress Test 2.3:}\nl
\begin{tabular}{|c||c|c||c|}
\hline 
\multicolumn{4}{|c|}{Computer Estimate}\\
\hline 
True Label & A24 & \aAuthor{A65} & Recall \\
\hline 
A24 &  &  &  0.00\\
\aAuthor{A65} & 1 &  &  0.0\\
\hline 
Precision & 0.0 & 0 & \\
\hline 
\multicolumn{4}{|c|}{Overall Accuracy: 0 Macro-average F-measure: 0}\\
\hline 
\end{tabular} 
\nl

\texttt{Random selection:}\nl
\begin{tabular}{|c|c|c|c|}
\hline 
Name & Recall &  Hits & Miss \\ 
\hline 
\aAuthor{A13$^{82}$} & 0.19 & 0.19 & 0.0 \\ 
\hline 
\multicolumn{4}{|c|}{Overall Accuracy: 0.19  Macro-average F-measure: 0.32 }\\ 
\hline 
\end{tabular}

\subsubsection{StressTest3}
One post from an author who has written \postAmount{Many} posts are checked against the entire corpus, \nref{StressTest4}.\\

\texttt{Stress Test 3.1:}\nl
\begin{tabular}{|c||c|c||c|}
\hline 
\multicolumn{4}{|c|}{Computer Estimate}\\
\hline 
True Label & A1$^{105}$ & \aAuthor{A35}$^{119}$ & Recall \\
\hline 
A1$^{105}$ &  &  &  0.0\\
\aAuthor{A35}$^{119}$ & 1 &  &  0.0\\
\hline 
Precision & 0.0 & 0.0 & \\
\hline 
\multicolumn{4}{|c|}{Overall Accuracy: 0 Macro-average F-measure: 0.0}\\
\hline 
\end{tabular} 
\nl

\texttt{Stress Test 3.2:}\nl
\begin{tabular}{|c||c|c||c|}
\hline 
\multicolumn{4}{|c|}{Computer Estimate}\\
\hline 
True Label & A1 & \aAuthor{A38} & Recall \\
\hline 
A1 &  &  &  0.00\\
\aAuthor{A38} & 1 &  &  0.0\\
\hline 
Precision & 0.0 & 0 & \\
\hline 
\multicolumn{4}{|c|}{Overall Accuracy: 0 Macro-average F-measure: 0}\\
\hline 
\end{tabular} 
\nl

\texttt{Stress Test 3.3:}\nl
\begin{tabular}{|c||c|c||c|}
\hline 
\multicolumn{4}{|c|}{Computer Estimate}\\
\hline 
True Label & A1 & \aAuthor{A65} & Recall \\
\hline 
A1 &  &  &  0.00\\
\aAuthor{A65} & 1 &  &  0.0\\
\hline 
Precision & 0.0 & 0 & \\
\hline 
\multicolumn{4}{|c|}{Overall Accuracy: 0 Macro-average F-measure: 0}\\
\hline 
\end{tabular} 
\nl

\texttt{Random selection:}\nl
\begin{tabular}{|c|c|c|c|c||c|c|c|c|c|}
\hline 
Name & Recall & Precision & Hits & Miss &Name & Recall & Precision & Hits & Miss \\ 
\hline 
A1$^{105}$ & 0.0 & 0 & 0 & 0.290111 & A4$^{137}$ & 0.0 & 0 & 0 & 0.379568 \\ 
\hline 
\multicolumn{10}{|c|}{Overall Accuracy: 0.33 Macro-average F-measure: 0.51}\\ 
\multicolumn{10}{|c|}{Total number of posts attributed to authors with less than 1 posts: 0}\\ 
\multicolumn{10}{|c|}{Percentage of posts attributed authors with 1 post: 0.0\%}\\ 
\hline 
\end{tabular}

\subsubsection{StressTest4}
A single text from an author is checked against all the texts, \nref{StressTest4} .\\

\texttt{Stress Test 4.1:}\nl
\begin{tabular}{|c||c|c||c|}
\hline 
\multicolumn{4}{|c|}{Computer Estimate}\\
\hline 
True Label & \veryFew{\aAuthor{A4}}$^{1}$ & \veryFew{A40}$^{1}$ & Recall \\
\hline 
\veryFew{\aAuthor{A4}}$^{1}$ &  & 1 &  0.0\\
\veryFew{A40}$^{1}$ &  &  &  0.0\\
\hline 
Precision & 0.0 & 0.0 & \\
\hline 
\multicolumn{4}{|c|}{Overall Accuracy: 0 Macro-average F-measure: 0.0}\\
\hline 
\end{tabular} 
\nl

\texttt{Stress Test 4.2:}\nl
\begin{tabular}{|c||c|c||c|}
\hline 
\multicolumn{4}{|c|}{Computer Estimate}\\
\hline 
True Label & \veryFew{\aAuthor{A38}}$^{1}$ & \veryFew{A65}$^{1}$ & Recall \\
\hline 
\veryFew{\aAuthor{A38}}$^{1}$ &  & 1 &  0.0\\
\veryFew{A65}$^{1}$ &  &  &  0.0\\
\hline 
Precision & 0.0 & 0.0 & \\
\hline 
\multicolumn{4}{|c|}{Overall Accuracy: 0 Macro-average F-measure: 0.0}\\
\hline 
\end{tabular} 
\nl

\texttt{Stress Test 4.3:}\nl
\begin{tabular}{|c||c|c||c|}
\hline 
\multicolumn{4}{|c|}{Computer Estimate}\\
\hline 
True Label & A32 & \aAuthor{A65} & Recall \\
\hline 
A32 &  &  &  0.00\\
\aAuthor{A65} & 1 &  &  0.0\\
\hline 
Precision & 0.0 & 0 & \\
\hline 
\multicolumn{4}{|c|}{Overall Accuracy: 0 Macro-average F-measure: 0}\\
\hline 
\end{tabular} 
\nl

\texttt{Random selection:}\nl
\begin{tabular}{|c|c|c|c|c||c|c|c|c|c|}
\hline 
Name & Recall & Precision & Hits & Miss &Name & Recall & Precision & Hits & Miss \\ 
\hline 
\veryFew{A0$^{1}$} & 0.0 & 0 & 0 & 0.011011 & \veryFew{A1$^{1}$} & 0.0 & 0 & 0 & 0.011087 \\ 
\hline 
\multicolumn{10}{|c|}{Overall Accuracy: 0.011 Macro-average F-measure: 0.022}\\ 
\multicolumn{10}{|c|}{Total number of posts attributed to authors with less than 1 posts: 0.022098}\\ 
\multicolumn{10}{|c|}{Percentage of posts attributed authors with 1 post: 100.0\%}\\ 
\hline 
\end{tabular}

\subsubsection{AuthorSomePost}
All the posts from a random author who has written \postAmount{Some} posts is checked against all the posts from all authors who have written \postAmount{Some} posts \nref{AuthorSomePost}.\\  

\texttt{AuthorSomePost 1:}\nl
\begin{tabular}{|c||c|c||c|}
\hline 
\multicolumn{4}{|c|}{Computer Estimate}\\
\hline 
True Label & \aAuthor{A0} & A24 & Recall \\
\hline 
\aAuthor{A0} &  & 75 &  0.0\\
A24 &  &  &  0.00\\
\hline 
Precision & 0 & 0.0 & \\
\hline 
\multicolumn{4}{|c|}{Overall Accuracy: 0 Macro-average F-measure: 0}\\
\hline 
\end{tabular} 
\nl

\texttt{AuthorSomePost 2:}\nl
\begin{tabular}{|c||c|c|c||c|}
\hline 
\multicolumn{5}{|c|}{Computer Estimate}\\
\hline 
True Label & \aAuthor{A2} & A24 & A3 & Recall \\
\hline 
\aAuthor{A2} &  & 97 & 1 &  0.0\\
A24 &  &  &  &  0.00\\
A3 &  &  &  &  0.00\\
\hline 
Precision & 0 & 0.0 & 0.0 & \\
\hline 
\multicolumn{5}{|c|}{Overall Accuracy: 0 Macro-average F-measure: 0}\\
\hline 
\end{tabular} 
\nl

\texttt{AuthorSomePost 3:}\nl
\begin{tabular}{|c||c|c|c||c|}
\hline 
\multicolumn{5}{|c|}{Computer Estimate}\\
\hline 
True Label & \aAuthor{A3}$^{99}$ & A13$^{82}$ & A24$^{71}$ & Recall \\
\hline 
\aAuthor{A3}$^{99}$ & 1 & 1 & 97 &  0.010\\
A13$^{82}$ &  &  &  &  0.0\\
A24$^{71}$ &  &  &  &  0.0\\
\hline 
Precision & 1 & 0.0 & 0.0 & \\
\hline 
\multicolumn{5}{|c|}{Overall Accuracy: 0.010 Macro-average F-measure: 0.02}\\
\hline 
\end{tabular} 
\nl

\texttt{Random selection:}\nl
\begin{tabular}{|c|c|c|c|c||c|c|c|c|c|}
\hline 
Name & Recall & Precision & Hits & Miss &Name & Recall & Precision & Hits & Miss \\ 
\hline 
\aAuthor{A0$^{75}$} & 0.18 & 1.0 & 13.241624 & 0.0 & A2$^{98}$ & 0.0 & 0 & 0 & 17.290265 \\ 
\hline 
\multicolumn{10}{|c|}{Overall Accuracy: 0.18 Macro-average F-measure: 0.3}\\ 
\multicolumn{10}{|c|}{Total number of posts attributed to authors with less than 1 posts: 0}\\ 
\multicolumn{10}{|c|}{Percentage of posts attributed authors with 1 post: 0.0\%}\\ 
\hline 
\end{tabular}

\subsubsection{AuthorManyPost}
All the posts from a random author who has written \postAmount{Many} posts is checked against all the posts from all authors who have written \postAmount{Many} posts \nref{AuthorManyPost}.\\  

\texttt{AuthorManyPost 1:}\nl
\begin{tabular}{|c||c|c|c||c|}
\hline 
\multicolumn{5}{|c|}{Computer Estimate}\\
\hline 
True Label & A1$^{105}$ & A4$^{137}$ & \aAuthor{A35}$^{119}$ & Recall \\
\hline 
A1$^{105}$ &  &  &  &  0.0\\
A4$^{137}$ &  &  &  &  0.0\\
\aAuthor{A35}$^{119}$ & 115 & 1 & 3 &  0.025\\
\hline 
Precision & 0.0 & 0.0 & 1 & \\
\hline 
\multicolumn{5}{|c|}{Overall Accuracy: 0.025 Macro-average F-measure: 0.05}\\
\hline 
\end{tabular} 
\nl

\texttt{AuthorManyPost 2:}\nl
\begin{tabular}{|c||c|c||c|}
\hline 
\multicolumn{4}{|c|}{Computer Estimate}\\
\hline 
True Label & A1$^{105}$ & \aAuthor{A4}$^{137}$ & Recall \\
\hline 
A1$^{105}$ &  &  &  0.0\\
\aAuthor{A4}$^{137}$ & 88 & 49 &  0.36\\
\hline 
Precision & 0.0 & 1 & \\
\hline 
\multicolumn{4}{|c|}{Overall Accuracy: 0.36 Macro-average F-measure: 0.51}\\
\hline 
\end{tabular} 
\nl

\texttt{AuthorManyPost 3:}\nl
\begin{tabular}{|c||c|c||c|}
\hline 
\multicolumn{4}{|c|}{Computer Estimate}\\
\hline 
True Label & \aAuthor{A1} & A4 & Recall \\
\hline 
\aAuthor{A1} & 101 & 4 &  0.96\\
A4 &  &  &  0.00\\
\hline 
Precision & 1.0 & 0.0 & \\
\hline 
\multicolumn{4}{|c|}{Overall Accuracy: 0.96 Macro-average F-measure: 0.95}\\
\hline 
\end{tabular} 
\nl

\texttt{Random selection:}\nl
\begin{tabular}{|c|c|c|c|c||c|c|c|c|c|}
\hline 
Name & Recall & Precision & Hits & Miss &Name & Recall & Precision & Hits & Miss \\ 
\hline 
A1$^{105}$ & 0.0 & 0 & 0 & 34.61188 & A4$^{137}$ & 0.0 & 0 & 0 & 45.156783 \\ 
\hline 
\multicolumn{10}{|c|}{Overall Accuracy: 0.33 Macro-average F-measure: 0.51}\\ 
\multicolumn{10}{|c|}{Total number of posts attributed to authors with less than 1 posts: 0}\\ 
\multicolumn{10}{|c|}{Percentage of posts attributed authors with 1 post: 0.0\%}\\ 
\hline 
\end{tabular}

\subsubsection{ShortBogusTest}
All texts from an author are tested against a corpus comprising only of the authors text, and 2 short bogus texts, in order to whether authors with few texts are given a too large probability when tested against. \pref{ShortBogusTest}

\texttt{Short Bogus Test 1:}\nl
\begin{tabular}{|c||c|c||c|}
\hline 
\multicolumn{4}{|c|}{Computer Estimate}\\
\hline 
True Label & A35 & Bogus & Recall \\
\hline 
A35 &  & 1 &  0.0\\
Bogus &  &  &  0.00\\
\hline 
Precision & 0.0 & 0.0 & \\
\hline 
\multicolumn{4}{|c|}{Overall Accuracy: 0.0 Macro-average F-measure: 0.0}\\
\hline 
\end{tabular} 
\nl

\texttt{Short Bogus Test 2:}\nl
\begin{tabular}{|c||c|c||c|}
\hline 
\multicolumn{4}{|c|}{Computer Estimate}\\
\hline 
True Label & A4 & Bogus & Recall \\
\hline 
A4 &  & 1 &  0.0\\
Bogus &  &  &  0.00\\
\hline 
Precision & 0.0 & 0.0 & \\
\hline 
\multicolumn{4}{|c|}{Overall Accuracy: 0.0 Macro-average F-measure: 0.0}\\
\hline 
\end{tabular} 
\nl

\texttt{Short Bogus Test 3:}\nl
\begin{tabular}{|c||c|c||c|}
\hline 
\multicolumn{4}{|c|}{Computer Estimate}\\
\hline 
True Label & \aAuthor{A1}$^{105}$ & Bogus$^{2}$ & Recall \\
\hline 
\aAuthor{A1}$^{105}$ &  & 1 &  0.0\\
Bogus$^{2}$ &  &  &  0.0\\
\hline 
Precision & 0.0 & 0.0 & \\
\hline 
\multicolumn{4}{|c|}{Overall Accuracy: 0 Macro-average F-measure: 0.0}\\
\hline 
\end{tabular} 
\nl

\texttt{Random selection:}\nl
\begin{tabular}{|c|c|c|c|}
\hline 
Name & Recall &  Hits & Miss \\ 
\hline 
\aAuthor{A35$^{119}$} & 0.98 & 0.98 & 0.0 \\ 
\hline 
\multicolumn{4}{|c|}{Overall Accuracy: 0.98  Macro-average F-measure: 1.0 }\\ 
\hline 
\end{tabular}

\subsubsection{UltimateStressTest}
A random half of the corpus is checked against the entire corpus, \nref{UltimateStressTest}.\\
Due to the size of the table, I have chosen to not represent the table in the manner used in \cite{nr4}, but instead just summing up the recall, precision, hits, misses (the number texts wrongly attributed the author), the number and percentage of texts attributed to an author with only 1 post.
\clearpage
\texttt{Ultimate Test 1:}\nl
\begin{tabular}{|c||c|c|c|c|c|c|c|c|c|c|c|c|c|c|c|c|c|c|c|c|c|c|c|c|c|c|c|c|c|c|c|c|c|c|c|c|c|c|c|c|c|c|c|c|c|c|c|c|c|c||c|}
\hline 
\multicolumn{52}{|c|}{Computer Estimate}\\
\hline 
True Label & \aAuthor{A0} & \aAuthor{A10} & \aAuthor{A11} & \aAuthor{A12} & \aAuthor{A13} & \aAuthor{A14} & \aAuthor{A15} & \aAuthor{A16} & \aAuthor{A17} & \aAuthor{A20} & \aAuthor{A26} & \aAuthor{A3} & \aAuthor{A30} & \aAuthor{A33} & \aAuthor{A34} & \aAuthor{A35} & \aAuthor{A36} & \aAuthor{A4} & \aAuthor{A40} & \aAuthor{A44} & \aAuthor{A46} & \aAuthor{A49} & \aAuthor{A50} & \aAuthor{A51} & \aAuthor{A52} & \aAuthor{A57} & \aAuthor{A58} & \aAuthor{A6} & \aAuthor{A60} & A61 & A62 & \aAuthor{A65} & A66 & \aAuthor{A67} & A68 & \aAuthor{A69} & A7 & \aAuthor{A72} & \aAuthor{A73} & \aAuthor{A74} & \aAuthor{A75} & \aAuthor{A77} & \aAuthor{A78} & A79 & \aAuthor{A80} & \aAuthor{A81} & \aAuthor{A83} & \aAuthor{A86} & A88 & \aAuthor{A89} & Recall \\
\hline 
\aAuthor{A0} &  &  &  &  &  &  &  &  &  &  &  &  &  &  &  &  & 7 &  &  &  &  &  & 6 &  &  &  &  &  &  & 51 &  &  & 5 &  & 1 &  &  &  &  &  &  &  &  &  &  &  &  &  & 1 & 4 &  0.0\\
\aAuthor{A10} &  &  &  &  &  &  &  &  &  &  &  &  &  &  &  &  & 1 &  &  &  &  &  & 1 &  &  &  &  &  &  & 2 &  &  &  &  &  &  &  &  &  &  &  &  &  &  &  &  &  &  &  &  &  0.0\\
\aAuthor{A11} &  &  &  &  &  &  &  &  &  &  &  &  &  &  &  &  & 1 &  &  &  &  &  &  &  &  &  &  &  &  & 4 &  &  &  &  & 1 &  &  &  &  &  &  &  &  &  &  &  &  &  &  &  &  0.0\\
\aAuthor{A12} &  &  &  &  &  &  &  &  &  &  &  &  &  &  &  &  & 2 &  &  &  &  &  &  &  &  &  &  &  &  & 5 &  &  & 3 &  &  &  &  &  &  &  &  &  & 1 &  &  &  &  &  & 1 &  &  0.0\\
\aAuthor{A13} &  &  &  &  &  &  &  &  &  &  &  &  &  &  &  &  & 11 &  &  &  &  &  & 4 &  &  &  &  &  &  & 37 & 1 &  & 9 &  & 7 &  &  &  & 2 &  &  &  &  &  &  &  &  &  & 5 & 6 &  0.0\\
\aAuthor{A14} &  &  &  &  &  &  &  &  &  &  &  &  &  &  &  &  &  &  &  &  &  &  &  &  &  &  &  &  &  & 1 &  &  &  &  &  &  &  &  &  &  &  &  &  &  &  &  &  &  &  &  &  0.0\\
\aAuthor{A15} &  &  &  &  &  &  &  &  &  &  &  &  &  &  &  &  &  &  &  &  &  &  & 1 &  &  &  &  &  &  &  &  &  &  &  &  &  &  &  &  &  &  &  &  &  &  &  &  &  &  &  &  0.0\\
\aAuthor{A16} &  &  &  &  &  &  &  &  &  &  &  &  &  &  &  &  &  &  &  &  &  &  &  &  &  &  &  &  &  & 1 &  &  & 2 &  &  &  &  &  &  &  &  &  &  &  &  &  &  &  &  &  &  0.0\\
\aAuthor{A17} &  &  &  &  &  &  &  &  & 1 &  &  &  &  &  &  &  &  &  &  &  &  &  &  &  &  &  &  &  &  &  &  &  &  &  &  &  &  &  &  &  &  &  &  &  &  &  &  &  &  &  &  1\\
\aAuthor{A20} &  &  &  &  &  &  &  &  &  &  &  &  &  &  &  &  & 3 &  &  &  &  &  &  &  &  &  &  &  &  & 9 & 1 &  &  &  &  &  &  &  &  &  &  &  &  &  &  &  &  &  & 1 & 1 &  0.0\\
\aAuthor{A26} &  &  &  &  &  &  &  &  &  &  &  &  &  &  &  &  & 1 &  &  &  &  &  & 3 &  &  &  &  &  &  & 2 &  &  & 1 &  & 1 &  &  &  &  &  &  &  &  &  &  &  &  &  &  &  &  0.0\\
\aAuthor{A3} &  &  &  &  &  &  &  &  & 1 &  &  &  &  &  &  &  & 15 &  &  &  &  &  & 7 &  &  &  &  &  &  & 41 & 1 &  & 4 &  & 18 &  & 1 &  &  &  &  &  &  &  &  &  &  &  & 2 & 9 &  0.0\\
\aAuthor{A30} &  &  &  &  &  &  &  &  & 1 &  &  &  &  &  &  &  & 1 &  &  &  &  &  & 4 &  &  & 1 &  &  &  & 6 & 1 &  & 4 &  & 2 &  &  &  & 1 &  &  &  &  &  &  &  &  &  & 1 & 3 &  0.0\\
\aAuthor{A33} &  &  &  &  &  &  &  &  &  &  &  &  &  &  &  &  &  &  &  &  &  &  &  &  &  &  &  &  &  & 1 &  &  &  &  &  &  &  &  &  &  &  &  &  &  &  &  &  &  &  &  &  0.0\\
\aAuthor{A34} &  &  &  &  &  &  &  &  &  &  &  &  &  &  &  &  & 1 &  &  &  &  &  &  &  &  &  &  &  &  & 3 &  &  &  &  &  &  &  &  &  &  &  &  &  &  &  &  &  &  &  &  &  0.0\\
\aAuthor{A35} &  &  &  &  &  &  &  &  &  &  &  &  &  &  &  &  & 19 &  &  &  &  &  & 6 &  &  &  &  &  &  & 67 & 1 &  & 2 &  & 7 &  &  &  & 1 &  &  &  & 8 &  &  &  &  &  & 1 & 7 &  0.0\\
\aAuthor{A36} &  &  &  &  &  &  &  &  &  &  &  &  &  &  &  &  & 1 &  &  &  &  &  &  &  &  &  &  &  &  &  &  &  &  &  &  &  &  &  &  &  &  &  &  &  &  &  &  &  &  &  &  1\\
\aAuthor{A4} &  &  &  &  &  &  &  &  &  &  &  &  &  &  &  &  & 11 &  &  &  &  &  & 18 &  &  & 2 &  &  &  & 48 & 1 &  & 28 &  & 8 &  &  &  &  &  &  &  & 12 &  &  &  &  &  & 3 & 6 &  0.0\\
\aAuthor{A40} &  &  &  &  &  &  &  &  &  &  &  &  &  &  &  &  & 1 &  &  &  &  &  & 1 &  &  &  &  &  &  & 1 &  &  &  &  & 1 &  &  &  &  &  &  &  &  &  &  &  &  &  &  &  &  0.0\\
\aAuthor{A44} &  &  &  &  &  &  &  &  &  &  &  &  &  &  &  &  &  &  &  &  &  &  &  &  &  &  &  &  &  & 2 &  &  &  &  &  &  &  &  &  &  &  &  &  &  &  &  &  &  &  &  &  0.0\\
\aAuthor{A46} &  &  &  &  &  &  &  &  &  &  &  &  &  &  &  &  &  &  &  &  &  &  &  &  &  &  &  &  &  &  &  &  &  &  &  &  &  &  &  &  &  &  &  & 1 &  &  &  &  &  &  &  0.0\\
\aAuthor{A49} &  &  &  &  &  &  &  &  &  &  &  &  &  &  &  &  &  &  &  &  &  &  &  &  &  &  &  &  &  & 2 &  &  &  &  &  &  &  &  &  &  &  &  &  &  &  &  &  &  &  &  &  0.0\\
\aAuthor{A50} &  &  &  &  &  &  &  &  &  &  &  &  &  &  &  &  &  &  &  &  &  &  & 1 &  &  &  &  &  &  &  &  &  &  &  &  &  &  &  &  &  &  &  &  &  &  &  &  &  &  &  &  1\\
\aAuthor{A51} &  &  &  &  &  &  &  &  &  &  &  &  &  &  &  &  &  &  &  &  &  &  &  &  &  &  &  &  &  & 2 &  &  &  &  &  &  &  &  &  &  &  &  &  &  &  &  &  &  & 1 &  &  0.0\\
\aAuthor{A52} &  &  &  &  &  &  &  &  &  &  &  &  &  &  &  &  & 5 &  &  &  &  &  & 1 &  &  & 1 &  &  &  & 10 &  &  & 3 &  & 6 &  & 1 &  &  &  &  &  &  &  &  &  &  &  &  & 1 &  0.0\\
\aAuthor{A57} &  &  &  &  &  &  &  &  &  &  &  &  &  &  &  &  &  &  &  &  &  &  &  &  &  & 2 &  &  &  &  &  &  &  &  &  &  &  &  &  &  &  &  &  &  &  &  &  &  &  &  &  1\\
\aAuthor{A58} &  &  &  &  &  &  &  &  &  &  &  &  &  &  &  &  & 3 &  &  &  &  &  &  &  &  &  &  &  &  & 2 & 1 &  &  &  & 2 &  &  &  &  &  &  &  &  &  &  &  &  &  & 2 & 2 &  0.0\\
\aAuthor{A6} &  &  &  &  &  &  &  &  &  &  &  &  &  &  &  &  &  &  &  &  &  &  & 2 &  &  &  &  &  &  & 2 &  &  &  &  & 1 &  &  &  &  &  &  &  &  &  &  &  &  &  &  & 1 &  0.0\\
\aAuthor{A60} &  &  &  &  &  &  &  &  &  &  &  &  &  &  &  &  &  &  &  &  &  &  &  &  &  &  &  &  &  & 2 & 1 &  &  &  &  &  &  &  &  &  &  &  &  &  &  &  &  &  &  &  &  0.0\\
A61 &  &  &  &  &  &  &  &  &  &  &  &  &  &  &  &  &  &  &  &  &  &  &  &  &  &  &  &  &  &  &  &  &  &  &  &  &  &  &  &  &  &  &  &  &  &  &  &  &  &  &  0.00\\
A62 &  &  &  &  &  &  &  &  &  &  &  &  &  &  &  &  &  &  &  &  &  &  &  &  &  &  &  &  &  &  &  &  &  &  &  &  &  &  &  &  &  &  &  &  &  &  &  &  &  &  &  0.00\\
\aAuthor{A65} &  &  &  &  &  &  &  &  &  &  &  &  &  &  &  &  & 2 &  &  &  &  &  &  &  &  &  &  &  &  & 5 &  &  &  &  & 3 &  &  &  &  &  &  &  &  &  &  &  &  &  &  &  &  0.0\\
A66 &  &  &  &  &  &  &  &  &  &  &  &  &  &  &  &  &  &  &  &  &  &  &  &  &  &  &  &  &  &  &  &  &  &  &  &  &  &  &  &  &  &  &  &  &  &  &  &  &  &  &  0.00\\
\aAuthor{A67} &  &  &  &  &  &  &  &  &  &  &  &  &  &  &  &  &  &  &  &  &  &  &  &  &  &  &  &  &  &  &  &  &  &  &  &  &  &  &  &  &  &  &  &  &  &  &  &  & 2 &  &  0.0\\
A68 &  &  &  &  &  &  &  &  &  &  &  &  &  &  &  &  &  &  &  &  &  &  &  &  &  &  &  &  &  &  &  &  &  &  &  &  &  &  &  &  &  &  &  &  &  &  &  &  &  &  &  0.00\\
\aAuthor{A69} &  &  &  &  &  &  &  &  &  &  &  &  &  &  &  &  & 1 &  &  &  &  &  &  &  &  &  &  &  &  & 3 & 1 &  &  &  &  &  &  &  &  &  &  &  &  &  &  &  &  &  &  &  &  0.0\\
A7 &  &  &  &  &  &  &  &  &  &  &  &  &  &  &  &  &  &  &  &  &  &  &  &  &  &  &  &  &  &  &  &  &  &  &  &  &  &  &  &  &  &  &  &  &  &  &  &  &  &  &  0.00\\
\aAuthor{A72} &  &  &  &  &  &  &  &  &  &  &  &  &  &  &  &  &  &  &  &  &  &  &  &  &  &  &  &  &  &  & 1 &  &  &  &  &  &  &  &  &  &  &  &  &  &  &  &  &  &  &  &  0.0\\
\aAuthor{A73} &  &  &  &  &  &  &  &  &  &  &  &  &  &  &  &  &  &  &  &  &  &  &  &  &  &  &  &  &  &  &  &  &  &  &  &  &  &  & 1 &  &  &  &  &  &  &  &  &  &  &  &  1\\
\aAuthor{A74} &  &  &  &  &  &  &  &  &  &  &  &  &  &  &  &  &  &  &  &  &  &  &  &  &  &  &  &  &  & 1 &  &  &  &  &  &  &  &  &  &  &  &  &  &  &  &  &  &  &  &  &  0.0\\
\aAuthor{A75} &  &  &  &  &  &  &  &  &  &  &  &  &  &  &  &  & 2 &  &  &  &  &  &  &  &  &  &  &  &  & 7 &  &  & 1 &  & 2 &  &  &  &  &  &  &  &  &  &  &  &  &  &  &  &  0.0\\
\aAuthor{A77} &  &  &  &  &  &  &  &  &  &  &  &  &  &  &  &  &  &  &  &  &  &  &  &  &  &  &  &  &  &  &  &  &  &  &  &  &  &  &  &  &  & 1 &  &  &  &  &  &  &  &  &  1\\
\aAuthor{A78} &  &  &  &  &  &  &  &  &  &  &  &  &  &  &  &  &  &  &  &  &  &  &  &  &  &  &  &  &  &  &  &  &  &  &  &  &  &  &  &  &  &  & 1 &  &  &  &  &  &  &  &  1\\
A79 &  &  &  &  &  &  &  &  &  &  &  &  &  &  &  &  &  &  &  &  &  &  &  &  &  &  &  &  &  &  &  &  &  &  &  &  &  &  &  &  &  &  &  &  &  &  &  &  &  &  &  0.00\\
\aAuthor{A80} &  &  &  &  &  &  &  &  &  &  &  &  &  &  &  &  &  &  &  &  &  &  &  &  &  &  &  &  &  &  &  &  & 1 &  &  &  &  &  &  &  &  &  &  &  &  &  &  &  &  &  &  0.0\\
\aAuthor{A81} &  &  &  &  &  &  &  &  &  &  &  &  &  &  &  &  &  &  &  &  &  &  &  &  &  &  &  &  &  &  &  &  & 1 &  &  &  &  &  &  &  &  &  &  &  &  &  &  &  &  &  &  0.0\\
\aAuthor{A83} &  &  &  &  &  &  &  &  &  &  &  &  &  &  &  &  &  &  &  &  &  &  &  &  &  &  &  &  &  &  &  &  &  &  &  &  &  &  &  &  &  &  &  &  &  &  & 1 &  &  &  &  1\\
\aAuthor{A86} &  &  &  &  &  &  &  &  &  &  &  &  &  &  &  &  & 1 &  &  &  &  &  &  &  &  &  &  &  &  &  &  &  &  &  &  &  &  &  &  &  &  &  &  &  &  &  &  &  &  &  &  0.0\\
A88 &  &  &  &  &  &  &  &  &  &  &  &  &  &  &  &  &  &  &  &  &  &  &  &  &  &  &  &  &  &  &  &  &  &  &  &  &  &  &  &  &  &  &  &  &  &  &  &  &  &  &  0.00\\
\aAuthor{A89} &  &  &  &  &  &  &  &  &  &  &  &  &  &  &  &  &  &  &  &  &  &  &  &  &  &  &  &  &  &  &  &  &  &  &  &  &  &  &  &  &  &  &  &  &  &  &  &  &  & 1 &  1\\
\hline 
Precision & 0 & 0 & 0 & 0 & 0 & 0 & 0 & 0 & 0.33 & 0 & 0 & 0 & 0 & 0 & 0 & 0 & 0.011 & 0 & 0 & 0 & 0 & 0 & 0.018 & 0 & 0 & 0.33 & 0 & 0 & 0 & 0.0 & 0.0 & 0 & 0.0 & 0 & 0.0 & 0 & 0.0 & 0 & 0.2 & 0 & 0 & 1 & 0.045 & 0.0 & 0 & 0 & 1 & 0 & 0.0 & 0.024 & \\
\hline 
\multicolumn{52}{|c|}{Overall Accuracy: 0.014 Macro-average F-measure: 0.081}\\
\hline 
\end{tabular} 


\clearpage
\texttt{Ultimate Test 2:}\nl
\begin{tabular}{|c|c|c|c|c||c|c|c|c|c|}
\hline 
Name & Recall & Precision & Hits & Miss &Name & Recall & Precision & Hits & Miss \\ 
\hline 
\aAuthor{A0$^{75}$} & 0.0 & 0.0 & 0 & 0 & \aAuthor{A3$^{99}$} & 0.0 & 0.0 & 0 & 0 \\ 
\hline 
\aAuthor{A4$^{137}$} & 0.0 & 0.0 & 0 & 0 & \aAuthor{A5$^{13}$} & 0.0 & 0.0 & 0 & 0 \\ 
\hline 
\veryFew{A7$^{1}$} & 0.0 & 0 & 0 & 4 & \aAuthor{A9$^{18}$} & 0.0 & 0.0 & 0 & 0 \\ 
\hline 
\aAuthor{A11$^{6}$} & 0.0 & 0.0 & 0 & 0 & \aAuthor{\veryFew{A15$^{1}$}} & 1 & 1 & 1 & 0 \\ 
\hline 
\aAuthor{A16$^{3}$} & 0.0 & 0.0 & 0 & 0 & \veryFew{A17$^{1}$} & 0.0 & 0 & 0 & 4 \\ 
\hline 
\aAuthor{A18$^{26}$} & 0.0 & 0.0 & 0 & 0 & \aAuthor{A19$^{27}$} & 0.0 & 0.0 & 0 & 0 \\ 
\hline 
\aAuthor{\veryFew{A22$^{1}$}} & 0.0 & 0.0 & 0 & 0 & \aAuthor{A23$^{2}$} & 0.0 & 0.0 & 0 & 0 \\ 
\hline 
\aAuthor{A25$^{15}$} & 0.0 & 0.0 & 0 & 0 & \aAuthor{A26$^{8}$} & 0.0 & 0.0 & 0 & 0 \\ 
\hline 
\aAuthor{A30$^{25}$} & 0.0 & 0.0 & 0 & 0 & \aAuthor{A35$^{119}$} & 0.0 & 0.0 & 0 & 0 \\ 
\hline 
\veryFew{A36$^{1}$} & 0.0 & 0 & 0 & 21 & \aAuthor{A38$^{5}$} & 0.0 & 0.0 & 0 & 0 \\ 
\hline 
\aAuthor{A39$^{2}$} & 0.0 & 0.0 & 0 & 0 & \aAuthor{A40$^{4}$} & 0.0 & 0.0 & 0 & 0 \\ 
\hline 
\aAuthor{A43$^{4}$} & 0.0 & 0.0 & 0 & 0 & \aAuthor{A44$^{2}$} & 0.0 & 0E+1 & 0 & 1 \\ 
\hline 
\aAuthor{\veryFew{A45$^{1}$}} & 1 & 0.5 & 1 & 1 & \aAuthor{A48$^{9}$} & 0.0 & 0.0 & 0 & 0 \\ 
\hline 
\aAuthor{A49$^{2}$} & 0.0 & 0.0 & 0 & 0 & \aAuthor{\veryFew{A50$^{1}$}} & 1 & 0.005 & 1 & 187 \\ 
\hline 
\aAuthor{A51$^{3}$} & 0.0 & 0.0 & 0 & 0 & \aAuthor{A53$^{7}$} & 0.0 & 0.0 & 0 & 0 \\ 
\hline 
A57$^{2}$ & 0.0 & 0 & 0 & 6 & \aAuthor{A58$^{12}$} & 0.0 & 0.0 & 0 & 0 \\ 
\hline 
\aAuthor{A60$^{3}$} & 0.0 & 0.0 & 0 & 0 & \veryFew{A61$^{1}$} & 0.0 & 0 & 0 & 150 \\ 
\hline 
\aAuthor{\veryFew{A62$^{1}$}} & 1 & 0.33 & 1 & 2 & \aAuthor{A63$^{4}$} & 0.0 & 0.0 & 0 & 0 \\ 
\hline 
\aAuthor{A65$^{10}$} & 0.0 & 0.0 & 0 & 0 & \veryFew{A66$^{1}$} & 0.0 & 0 & 0 & 82 \\ 
\hline 
\aAuthor{A67$^{2}$} & 0.5 & 0.045 & 1 & 21 & \aAuthor{\veryFew{A68$^{1}$}} & 1 & 0.012 & 1 & 85 \\ 
\hline 
\aAuthor{A69$^{5}$} & 0.0 & 0.0 & 0 & 0 & \aAuthor{\veryFew{A71$^{1}$}} & 1 & 1 & 1 & 0 \\ 
\hline 
\aAuthor{\veryFew{A73$^{1}$}} & 1 & 0.2 & 1 & 4 & \aAuthor{\veryFew{A74$^{1}$}} & 0.0 & 0.0 & 0 & 0 \\ 
\hline 
\aAuthor{A75$^{12}$} & 0.0 & 0.0 & 0 & 0 & \veryFew{A77$^{1}$} & 0.0 & 0 & 0 & 1 \\ 
\hline 
\multicolumn{10}{|c|}{Overall Accuracy: 0.015 Macro-average F-measure: 0.12}\\ 
\multicolumn{10}{|c|}{Total number of posts attributed to authors with less than 1 posts: 548}\\ 
\multicolumn{10}{|c|}{Percentage of posts attributed authors with 1 post: 94.97\%}\\ 
\hline 
\end{tabular}


\clearpage
\texttt{Ultimate Test 3:}\nl
\begin{tabular}{|c|c|c|c|c||c|c|c|c|c|}
\hline 
Name & Recall & Precision & Hits & Miss &Name & Recall & Precision & Hits & Miss \\ 
\hline 
\aAuthor{A1$^{105}$} & 0.0 & 0.0 & 0 & 0 & \aAuthor{A3$^{99}$} & 0.0 & 0.0 & 0 & 0 \\ 
\hline 
\aAuthor{A5$^{13}$} & 0.0 & 0.0 & 0 & 0 & \aAuthor{A6$^{6}$} & 0.0 & 0.0 & 0 & 0 \\ 
\hline 
\veryFew{A7$^{1}$} & 0.0 & 0 & 0 & 6 & \aAuthor{A8$^{5}$} & 0.0 & 0.0 & 0 & 0 \\ 
\hline 
\aAuthor{A9$^{18}$} & 0.0 & 0.0 & 0 & 0 & \aAuthor{A12$^{12}$} & 0.0 & 0.0 & 0 & 0 \\ 
\hline 
\aAuthor{A13$^{82}$} & 0.0 & 0.0 & 0 & 0 & \aAuthor{\veryFew{A15$^{1}$}} & 1 & 1 & 1 & 0 \\ 
\hline 
\aAuthor{\veryFew{A17$^{1}$}} & 1 & 0.2 & 1 & 4 & \aAuthor{A18$^{26}$} & 0.0 & 0.0 & 0 & 0 \\ 
\hline 
\aAuthor{A20$^{15}$} & 0.0 & 0.0 & 0 & 0 & \aAuthor{A21$^{22}$} & 0.0 & 0.0 & 0 & 0 \\ 
\hline 
\aAuthor{A24$^{71}$} & 0.0 & 0.0 & 0 & 0 & \aAuthor{A26$^{8}$} & 0.0 & 0.0 & 0 & 0 \\ 
\hline 
\aAuthor{\veryFew{A27$^{1}$}} & 0.0 & 0.0 & 0 & 0 & \aAuthor{A32$^{20}$} & 0.0 & 0.0 & 0 & 0 \\ 
\hline 
\veryFew{A36$^{1}$} & 0.0 & 0 & 0 & 26 & \aAuthor{A37$^{5}$} & 0.2 & 1 & 1 & 0 \\ 
\hline 
\aAuthor{A40$^{4}$} & 0.0 & 0.0 & 0 & 0 & \aAuthor{A42$^{22}$} & 0.0 & 0.0 & 0 & 0 \\ 
\hline 
A44$^{2}$ & 0.0 & 0 & 0 & 1 & \aAuthor{\veryFew{A45$^{1}$}} & 1 & 1 & 1 & 0 \\ 
\hline 
\aAuthor{A49$^{2}$} & 0.0 & 0.0 & 0 & 0 & \aAuthor{\veryFew{A50$^{1}$}} & 1 & 0.0045 & 1 & 213 \\ 
\hline 
\aAuthor{A51$^{3}$} & 0.0 & 0.0 & 0 & 0 & \aAuthor{A52$^{28}$} & 0.0 & 0.0 & 0 & 0 \\ 
\hline 
\aAuthor{A53$^{7}$} & 0.0 & 0.0 & 0 & 0 & \aAuthor{A54$^{31}$} & 0.0 & 0.0 & 0 & 0 \\ 
\hline 
\aAuthor{\veryFew{A55$^{1}$}} & 0.0 & 0.0 & 0 & 0 & \aAuthor{A56$^{18}$} & 0.0 & 0.0 & 0 & 0 \\ 
\hline 
A57$^{2}$ & 0.0 & 0 & 0 & 4 & \aAuthor{A58$^{12}$} & 0.0 & 0.0 & 0 & 0 \\ 
\hline 
\veryFew{A61$^{1}$} & 0.0 & 0 & 0 & 135 & \veryFew{A62$^{1}$} & 0.0 & 0 & 0 & 3 \\ 
\hline 
\aAuthor{A63$^{4}$} & 0.0 & 0.0 & 0 & 0 & \aAuthor{\veryFew{A66$^{1}$}} & 1 & 0.016 & 1 & 62 \\ 
\hline 
A67$^{2}$ & 0.0 & 0 & 0 & 25 & \aAuthor{\veryFew{A68$^{1}$}} & 1 & 0.012 & 1 & 81 \\ 
\hline 
\multicolumn{10}{|c|}{Overall Accuracy: 0.013 Macro-average F-measure: 0.12}\\ 
\multicolumn{10}{|c|}{Total number of posts attributed to authors with less than 1 posts: 536}\\ 
\multicolumn{10}{|c|}{Percentage of posts attributed authors with 1 post: 94.532627866\%}\\ 
\hline 
\end{tabular}

\clearpage
\texttt{Random selection:}\nl
\begin{tabular}{|c|c|c|c|c||c|c|c|c|c|}
\hline 
Name & Recall & Precision & Hits & Miss &Name & Recall & Precision & Hits & Miss \\ 
\hline 
\aAuthor{A0$^{75}$} & 0.05 & 0.11 & 4.23 & 35.09 & \aAuthor{A3$^{99}$} & 0.07 & 0.15 & 7.38 & 44.55 \\ 
\hline 
\aAuthor{A4$^{137}$} & 0.10 & 0.20 & 14.1 & 57.72 & \aAuthor{A6$^{6}$} & 0.00 & 0.00 & 0.02 & 3.117 \\ 
\hline 
\aAuthor{A10$^{4}$} & 0.00 & 0.00 & 0.01 & 2.088 & \aAuthor{A11$^{6}$} & 0.00 & 0.00 & 0.02 & 3.122 \\ 
\hline 
\aAuthor{A12$^{12}$} & 0.00 & 0.01 & 0.10 & 6.184 & \aAuthor{A13$^{82}$} & 0.06 & 0.12 & 5.05 & 37.95 \\ 
\hline 
\aAuthor{\veryFew{A14$^{1}$}} & 0.00 & 0.00 & 0.00 & 0.521 & \aAuthor{\veryFew{A15$^{1}$}} & 0.00 & 0.00 & 0.00 & 0.522 \\ 
\hline 
\aAuthor{A16$^{3}$} & 0.00 & 0.00 & 0.00 & 1.565 & \aAuthor{\veryFew{A17$^{1}$}} & 0.00 & 0.00 & 0.00 & 0.523 \\ 
\hline 
\aAuthor{A20$^{15}$} & 0.01 & 0.02 & 0.16 & 7.697 & \aAuthor{A26$^{8}$} & 0.00 & 0.01 & 0.04 & 4.146 \\ 
\hline 
\aAuthor{A30$^{25}$} & 0.01 & 0.03 & 0.46 & 12.64 & \aAuthor{\veryFew{A33$^{1}$}} & 0.00 & 0.00 & 0.00 & 0.524 \\ 
\hline 
\aAuthor{A34$^{4}$} & 0.00 & 0.00 & 0.01 & 2.085 & \aAuthor{A35$^{119}$} & 0.09 & 0.18 & 10.6 & 51.74 \\ 
\hline 
\aAuthor{\veryFew{A36$^{1}$}} & 0.00 & 0.00 & 0.00 & 0.522 & \aAuthor{A40$^{4}$} & 0.00 & 0.00 & 0.01 & 2.086 \\ 
\hline 
\aAuthor{A44$^{2}$} & 0.00 & 0.00 & 0.00 & 1.045 & \aAuthor{\veryFew{A46$^{1}$}} & 0.00 & 0.00 & 0.00 & 0.524 \\ 
\hline 
\aAuthor{A49$^{2}$} & 0.00 & 0.00 & 0.00 & 1.046 & \aAuthor{\veryFew{A50$^{1}$}} & 0.00 & 0.00 & 0.00 & 0.523 \\ 
\hline 
\aAuthor{A51$^{3}$} & 0.00 & 0.00 & 0.00 & 1.564 & \aAuthor{A52$^{28}$} & 0.02 & 0.04 & 0.58 & 14.08 \\ 
\hline 
\aAuthor{A57$^{2}$} & 0.00 & 0.00 & 0.00 & 1.045 & \aAuthor{A58$^{12}$} & 0.00 & 0.01 & 0.10 & 6.181 \\ 
\hline 
\aAuthor{A60$^{3}$} & 0.00 & 0.00 & 0.00 & 1.565 & \aAuthor{A65$^{10}$} & 0.00 & 0.01 & 0.07 & 5.172 \\ 
\hline 
\aAuthor{A67$^{2}$} & 0.00 & 0.00 & 0.00 & 1.044 & \aAuthor{A69$^{5}$} & 0.00 & 0.00 & 0.01 & 2.606 \\ 
\hline 
\aAuthor{\veryFew{A72$^{1}$}} & 0.00 & 0.00 & 0.00 & 0.524 & \aAuthor{\veryFew{A73$^{1}$}} & 0.00 & 0.00 & 0.00 & 0.523 \\ 
\hline 
\aAuthor{\veryFew{A74$^{1}$}} & 0.00 & 0.00 & 0.00 & 0.523 & \aAuthor{A75$^{12}$} & 0.00 & 0.01 & 0.10 & 6.183 \\ 
\hline 
\aAuthor{\veryFew{A77$^{1}$}} & 0.00 & 0.00 & 0.00 & 0.523 & \aAuthor{\veryFew{A78$^{1}$}} & 0.00 & 0.00 & 0.00 & 0.524 \\ 
\hline 
\aAuthor{\veryFew{A80$^{1}$}} & 0.00 & 0.00 & 0.00 & 0.523 & \aAuthor{\veryFew{A81$^{1}$}} & 0.00 & 0.00 & 0.00 & 0.523 \\ 
\hline 
\aAuthor{\veryFew{A83$^{1}$}} & 0.00 & 0.00 & 0.00 & 0.522 & \aAuthor{\veryFew{A86$^{1}$}} & 0.00 & 0.00 & 0.00 & 0.522 \\ 
\hline 
\aAuthor{\veryFew{A89$^{1}$}} & 0.00 & 0.00 & 0.00 & 0.523 \\ 
\hline 
\multicolumn{10}{|c|}{Overall Accuracy: 0.062  Macro-average F-measure: 0.016 }\\ 
\multicolumn{10}{|c|}{Total number of posts attributed to authors with less than 1 posts: 8.911262}\\ 
\multicolumn{10}{|c|}{Percentage of posts attributed authors with 1 post: 2.437 \%}\\ 
\hline 
\end{tabular}
