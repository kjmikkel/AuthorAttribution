\section{Considerations}
\label{considerations}

In this section I will describe the different tests that I will use to gauge the effectiveness of the algorithm. In order to do so probably I will perform each test multiple times (with multiple different texts) in order to gain a better overview of how effective the process is. Unless specifically noted I will run each test 3 times. \fixme{Should I test more times? Or should I perhaps mention the times I want to test locally at the different tests?}

\subsection{Initial smoke tests}
\label{smokeTest}
Some of the initial smoke tests showed some problems with the algorithm. 
One of the problems I found was that some shorter, bogus texts, would be given the greatest probability, even when the text in question was present in the corps that was analysed. A possible explanation for this is that the shorter text, has a greater probability for each word (and thus also for the \fixme{is this entire correct?} words that does not appear - as per Good-Turing smoothing).

\subsection{Explanation of test description}
\test{
The name of the test
}{
The description of how the test will be done - including which kind of post that will be used in the corpus, and which kinds of posts that will be used for the author. 
}{
The purpose of the test - what are we to archive by doing it, and how likely it is to succeed, based on previous smoke-tests and the code.
}{
The entries whose author we are trying to attribute
}{
What the corpus is going to contain - how many entries, their size, etc. \ldots
}

\subsection{Values}
In order to make my results more independent of the dataset I will introduce the following values:

\subsubsection*{Number of posts}
$Axx_{\alpha} = $the number of posts author xx has written\\ 
$\Omega = max(A0_\alpha, A1_\alpha, \ldots A89_\alpha, A90_\alpha)$. In this case $\Omega = 137$\\

Let $\tau_x = \rm{ceil}(\Omega \times \frac{x}{4})$ and let the predicate $\#(a, \rm{num})$ be true if the author a has posted at least num papers. Let 

$$\Phi_x = \forall\rm{ author }\in\rm{ Authors, }\forall\rm{ posts }\in\rm{ author }: \{\rm{posts }\mid \#(a, \tau_x)\}$$ 
 
And then we have to check for $\Phi_1$ (\postAmount{few} = 35), $\Phi_2$ (\postAmount{some} = 69), $\Phi_3$ (\postAmount{many} = 103).

\subsubsection*{Length of posts}
\postSize{Short} post: A post that contains 100 characters or less\\
\postSize{Medium} post: A post that contains between 100 to 1000 characters\\
\postSize{Long} post: A post that contains between 1000 to 3000 characters\\
\postSize{Rant} post: A post of more than 3000 characters


\subsection{Selection of authors}
In order to select the authors I wrote a python script that found all of the authors in a selected corpus (for example only having texts from authors who have written a certain number of texts). It then randomly selects n of these, and saves it to a file, so that while the authors were randomly chosen, they values could be reused later. When the test are run, the post from these authors will be extracted from the corpus and tested one after the other.


\subsection{Tests}

\test{
StressTest1
}{
Take a post from an author appearing in the corpora with a single, \postSize{Long} text, and see if the author can be correctly attributed compared to the entire corpus.
}{
Check the limits of the algorithm. It will most likely not be able to attribute the author correctly.
}{
1 \postSize{Long} post
}{
The entire corpus.
}

\test{
StressTest2
}{
As above, but only against authors who have written \postAmount{Few} posts
}{
Check the limits of the algorithm. It will most likely not be able to attribute the author correctly.
}{
The same post as above
}{
All texts from authors who have written \postAmount{Few} posts
}

\test{
StressTest3
}{
As above, but only against authors who have written \postAmount{Some} posts
}{
To check the limits of the algorithm. This  limits of the algorithm. This has a chance of succeeding
}{
The same post as above
}{
All texts from authors who have written \postAmount{Some} posts
}

\test{
StressTest4
}{
As above, but only against authors who have written \postAmount{Some} posts
}{
Check the limits of the algorithm. This has a chance of succeeding
}{
The same post as above
}{
All texts from authors who have written \postAmount{Many} posts
}

\test{
StressTest5
}{
As above, but each author is only allowed 1 post.
}{
Check the limits of the algorithms. There is a non-trivial likelihood that this will succeed.
}{
The same post as above
}{
1 post from each author in the corpora
}

\test{ 
ShortBogusTest
}{
Take a post from an author A with multiple \postSize{Long} posts, and create a corpora from all the texts of A, and a short bogus post from a bogus author, and then try to see whether or not the bogus post is chosen
}{
A test to see whether the algorithm is susceptible to give a higher rank to posts with very little information (even if the bogus post is very dissimilar to the post that it is being checked against)
}{
1 long post from A 
}{
All the posts from A, plus 2 bogus post.
}

\test{
AuthorFewPost
}{
Take all the posts from an author A, where $\#(A, \tau_1)$ is true, and let the corpora be $\Phi_1$.
}{
A simple test to see whether the algorithm will work better texts that are longer, and have authors who have written several posts
}{
All posts from an author A, where $\#(A, \tau_1)$ is true
}{
$\Phi_1$.
}

\test{
AuthorSomePost
}{
Take all the posts from an author A, where $\#(A, \tau_2)$ is true, and let the corpora be $\Phi_2$.
}{
A simple test to see whether the algorithm will work better on texts that are longer, and have authors who have written some posts
}{
All posts from an author A, where $\#(A, \tau_2)$ is true
}{
$\Phi_2$.
}

\test{
AuthorManyPost
}{
Take all post from an author A, where $\#(A, \tau_3)$ is true, and let the corpora be $\Phi_3$
}{
A simple test to see whether the algorithm will work better on texts that are longer, and have authors who have written some posts
}{
All posts from an author A, where $\#(A, \tau_3)$ is true
}{
$\Phi_3$.
}

\test{
UltimateStressTest
}{
Take a random selection of authors, who have made over half the posts in the entire corpora, and test it on the rest of the corpora. Since they are choosen at random, some of the runs may have many authors, while some will only have few.
}{
This test is to check the overall quality of the algorithm - i.e. if it can attribute the posts correctly, based on the five attributes in the test section, and compared to the results found in \cite{nr4}.
}{
All the texts from random authors, who together have made over half the texts in the corpora
}{
The entire corpora
}
