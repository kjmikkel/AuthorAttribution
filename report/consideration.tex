\section{Considerations}
\label{considerations}

In this section I will describe the different tests that I will use to gauge the effectiveness of the algorithm. In order to do so probably, I will perform each test multiple times (with different author texts and corporas where applicable) in order to gain a better overview of the effectiveness of the algorithm. Unless otherwise noted, I  will run each test 3 times.

\subsection{Initial smoke tests}
\label{smokeTest}
In the initial smoke tests I detected the following problems with the algorithm:
\begin{enumerate}
\item In the first toy problem I tested, the algorithm was unable to correctly attribute the text, even though there were only 2 authors, who both had only a single text in the corpora - the first a sting of a's, the second a string of b's. Text was attributed to the wrong author.

\item Authors of short bogus texts (i.e. texts that was produced by mashing the keyboard) were given a far greater probability than they warranted, and their authors were often chosen as the most probable author. 
\end{enumerate}
 

\subsection{Explanation of test description}
\test{
The name of the test
}{
The description of how the test will be done - including which kind of post that will be used in the corpora, and which kinds of posts that will be tested. 
}{
The purpose of the test - what are we to archive by doing it, and how likely it is to succeed, based on previous smoke-tests and code.
}{
A description of the texts we are attempting to attribute
}{
What the corpus is going to contain - how many texts and the what the corpora is going to contain etc.
}

\subsection{Values}
In order to make my results more independent of my current datasat - I introduce the following values:

\subsubsection*{Number of texts}
$Axx_{\alpha} = $the number of texts author xx has written\\ 
$\Omega = max(A0_\alpha, A1_\alpha, \ldots A89_\alpha, A90_\alpha)$. In this case $\Omega = 137$\\

Let $\tau_x = \rm{ceil}(\Omega \times \frac{x}{4})$ and let the predicate $\#(author, \rm{num})$ be true if author a has posted num papers. Let 

$$\Phi_x = \forall\rm{ author }\in\rm{ Authors, }\forall\rm{ texts }\in\rm{ author }: \{\rm{texts }\mid \#(a, \tau_x)\}$$ 
 
And so we have have to check for 
\begin{description}
\item[$\Phi_1$ \postAmount{few}:] 35-68
\item[$\Phi_2$ \postAmount{some}:] 69-102
\item[$\Phi_3$ \postAmount{many}:] 103-9999.
\end{description}

\subsubsection*{Length of texts}

\begin{description}
\item[\postSize{Short} post:] A post that contains 100 characters or less
\item[\postSize{Medium} post:] A post that contains between 100 to 1000 characters
\item[\postSize{Long} post:] A post that contains between 1000 to 3000 characters
\item[\postSize{Rant} post:] A post of more than 3000 characters
\end{description}

\subsection{Selection of authors}
In the tests where I have had to select authors, I have used a small python script, that, with the built-in Random function, have selected the authors.

\subsection{Authordata}
A list of the authors and their relevant data can be found in \pref{reportTable}. This includes number of texts each author has written, average of characters written etc.

\subsection{Tests}
I will in this section describe the various texts that I will run. It should be noted that there are no StressTest for \postAmount{few}, since no author has written \postAmount{few} texts.\\ 

\test{StressTest1}{
Take a post from an author appearing in the corpora with a single, \postSize{Long} text, and see if the author can be correctly attributed compared to the entire corpus.
}{
Check the limits of the algorithm. It will most likely not be able to attribute the author correctly.
}{
1 \postSize{Long} post
}{
The entire corpora
}

\test{StressTest2}{
Check a single post from an author who has written \postAmount{Some} texts against all texts written by authors who have written \postAmount{Some} texts
}{
To check the limits of the algorithm. This should have a greater chance of succeeding than the one above, since the post is only checked against texts in its own size category.
}{
A single random post from the corpora described below
}{
All texts from authors who have written \postAmount{Some} texts
}

\test{StressTest3}{
Check a single post from an author who has written \postAmount{Many} texts against all texts written by authors who have written \postAmount{Many} texts
}{
Check the limits of the algorithm. This has a chance of succeeding
}{
A single post from the corpora described below
}{
All texts from authors who have written \postAmount{Many} texts
}

\test{StressTest4}{
I check a single post against a corpora that includes exactly 1 post from each author.
}{
Check the limits of the algorithms. There is a non-trivial likelihood that this will succeed. It should be noted that due to the selection criteria for this corpora, authors with a single post contributes far more to the corpora than normally.
}{
The same post as in StressTest1
}{
1 post from each author in the corpora
}

\test{ShortBogusTest}{
Take a post from an author A with multiple \postSize{Long} texts, and create a corpora from all the texts of A, and a short bogus post from a bogus author.
}{
This test checks whether the algorithm gives a higher probability to authors with very little information.
}{
1 long post from A 
}{
All the texts from A, plus 2 bogus post.
}

\test{AuthorSomePost}{
Take all the texts from an author A, where $\#(A, \tau_2)$ is true, and let the corpora be $\Phi_2$.
}{
A simple test to see whether the algorithm will work better if all authors have written a comparatively equal number of texts, and there are \postAmount{Some} of them.
}{
All texts from an author A, where $\#(A, \tau_2)$ is true
}{
$\Phi_2$.
}

\test{AuthorManyPost}{
Take all post from an author A, where $\#(A, \tau_3)$ is true, and let the corpora be $\Phi_3$
}{
A simple test to see whether the algorithm will work better if all authors have written a comparatively equal number of texts, and there are \postAmount{Many} of them.
}{
All texts from an author A, where $\#(A, \tau_3)$ is true
}{
$\Phi_3$.
}x

\test{UltimateStressTest}{
Take a random selection of authors, who have made over half the texts in the entire corpora, and test it on the corpora. 
}{
This test is to check the overall quality of the algorithm - i.e. if it can attribute the texts correctly, based on the five attributes in the test section, and compared to the results found in \cite{nr4}.
}{
All the texts from random authors, who together have made over half the texts in the corpora
}{
The entire corpora
}
