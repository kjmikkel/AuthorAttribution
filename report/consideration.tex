\section{Considerations}
\label{considerations}

In this section I will describe the different tests that I will use to gauge the effectiveness of the algorithm. In order to do so probably I will perform each test multiple times (with multiple different texts) in order to gain a better overview of how effective the process is. Unless specifically noted I will run each test 3 times. \fixme{Should I test more times? Or should I perhaps mention the times I want to test locally at the different tests?}

\subsection{Initial smoke tests}
\label{smokeTest}
Some of the initial smoke tests showed some problems with the algorithm. 
One of the problems I found was that some shorter, bogus texts, would be given the greatest probability, even when the text in question was present in the corps that was analysed. A possible explanation for this is that the shorter text, has a greater probability for each word (and thus also for the \fixme{is this entire correct?} words that does not appear - as per Good-Turing smoothing).

\subsection{Limitations of the tests}
%In this study I have not been able to find texts from any of the authors from outside the source material that I have been working on. Therefore the only way

\test{
StressTest1
}{
Take a post from an author appearing in the corpora with a single, long text, and see if the author can be correctly attributed compared to the rest of the corpora.
}{
Check the limits of the algorithm. This is likely to fail.
}

\test{
StressTest2
}{
As above, but only against 100 posts
}{
Check the limits of the algorithm. This is likely to fail
}

\test{
StressTest3
}{
As above, but only against 10 posts
}{
Check the limits of the algorithm. This has a chance of succeding
}

\test{
StressTest4
}{
As above, but each author is only allowed one post.
}{
Check the limits of the algorithms. There is a non-trivial likelyhood that this will succeed.
}

\test{
ShortBogusTest
}{
Take a post from an author A with multiple long posts, and create a corpora with a short bogus post from a bogus author, together with real posts from A and others, and then try to see whether or not the bogus post is chosen
}{
A test to see whether the algorithm is suseptiable to give a heigher rank to posts with very little information (even if the bogus post is very dissimilar to the post that it is being cheked against)
}

\test{
Author10Post
}{
Take a post from an author A, who have written over 10 posts, and create a corpora with multiple long texts, including texts from author A. All authors must have written over 10 posts.
}{
A simple test to see whether the algorithm will work better on texts that are longer, and have authors who have written several posts
}

\test{
Author50Post
}{
Take a post from an author A, who have written over 50 posts, and create a corpora with multiple long texts, including texts from author A. All authors must have written over 50 posts.
}{
A simple test to see whether the algorithm will work better on texts that are longer, and have authors who have written many posts
}

\test{
Author100Post
}{
Take a post from an author A, who have written over 100 posts, and create a corpora with multiple long texts, including texts from author A. All authors must have written over 100 posts.
}{
A simple test to see whether the algorithm will work better on texts that are longer, and have authors who have written many posts
}


\test{

}{
Take a 
}{


}
