
% [Conclusions are very important. Do not expect that the reader remembers everything you told him/her.
% Having stated the definitions, you can now be more specific that  in the introduction]
% * Overview what this work was about.
% * Main results and contributions
% * Comments on importance or
% * Tips for practical use [how your results or experience can help someone in practice or
%     another researcher to use your simulator or avoid pitfalls]
% * Future work. Reinforce the importance of work, but avoid giving out your ideas].

\section{Conclusion}
\label{conclusion}

\subsection{Summery}
I have in this report investigated possible methods for Authorship Attribution for Internet forums, and discussed the advantages and disadvantages of each of these methods. I have come to the conclusion, that Internet forums should be the primary concern for Authorship Attribution, and that less work should be used on comment sections (such as YouTube) and Wikis.  I have given a detailed explanation of how the $n$-gram method proposed in \cite{nr4} works. 

\subsection{Results}
From section \ref{interpretation} I can conclude that the algorithm described in \cite{nr4} cannot be used with the Good-Turing smoothing for identifying user on Internet forums. The results in section \ref{tests} that when authors with few $n$-grams (i.e. has written few short posts) appear in the corpus, there is a high probability that they will be attributed as the author. It is therefore important, that future $n$-gram based Authorship Attribution methods for Internet forums, must give texts a more proportionate probability based on both their $n$-grams and length. If this is not done, then $n$-gram based methods will be of little use, as it will exclude copora with widely different number (and sizes) of texts --- which makes the method of little use for Authorship Attribution on Internet forums. 

\subsection{Future work}
\begin{itemize}
\item Since it is likely that many of the wrong attributions are caused by the way that Good-Turing smoothing assigns a far too high probability for the $n$-grams that does not appear in small corporas, it would be interesting to see whether using another smoothing technique (\cite{nr4} also proposes Witten-Bell smoothing) would improve the results.

\item While this report has produced bad results when it comes to texts written in the Danish language, it would be interesting to see the results of using it on different languages --- such as English, Greek or Chinese.

\item This report was made using 3-grams with characters as the token/gram, and it would therefore be interesting to see the results of a different configuration --- such as using words as the token/gram, and trying the entire spectrum of 3-6 grams. 

\item Attempt to use the a modified algorithm on a larger corpora --- preferably from an active forum with a large number of users. 

\item Turn a working algorithm into a plug-in for a forum, in order to field test its value in a real situation. Here it would especially be interesting to study the reactions from both forum users as well as the moderators. This should be done in order to study the social dimension of bringing Author Attribution to the Internet.
\end{itemize}
